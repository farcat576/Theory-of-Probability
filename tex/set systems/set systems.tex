\chapter{Set Systems}
\label{ch:set systems}

\section{Lecture 1 (August 12 2024)}
\begin{definition}[$\sigma$-algebra]
	Let $\Omega \neq \phi$ be an arbitrary set, then $\mathcal F$ is called a $\sigma$-algebra/$\sigma$-field if the following are true:
	\begin{enumerate}[i)]
		\item $\Omega \in \mathcal F$
		\item $A \in \mathcal F \implies A^C \in \mathcal F$
		\item $\forall \{A_n\}_{n \in \NN}$ such that $A_n \in \mathcal F$ $\forall n \in \NN$, $\cup_{n \in \NN} A_n \in \mathcal F$
	\end{enumerate}
\end{definition}

\begin{remark}
	If iii) only holds for finite unions, it is called a algebra/field.
\end{remark}

\begin{lemma}
	Let $\mathcal F \subseteq 2^\Omega$ be a $\sigma$-algebra, then:
	\begin{enumerate}[i)]
		\item For some $ n \in \NN$, $A_i \in \mathcal F$ for $1\leq i \leq n \implies \cup_{i=1}^n A_i \in \mathcal F$
		\item $\forall n \in \NN$, $A_n \in \mathcal F \implies \cap_{n \in \NN} A_n \in \mathcal F$
		\item $A,B \in \mathcal F \implies A \setminus B \in \mathcal F$
	\end{enumerate}
\end{lemma}

\begin{proof}
	\begin{enumerate}[i)]
		\item Define $\{B_k\}_{k \in \NN}$ such that $B_k=\begin{cases}
			A_k & 1\leq k \leq n \\
			\phi & k>n 
			\end{cases}$ (Since $\Omega \in \mathcal F$, $\Omega^C=\phi \in \mathcal F$) \\
			We are done as $\cup_{k \in \NN} B_k=\cup_{i=1}^n A_i \in \mathcal F$
		\item Define $\{B_k\}_{k \in \NN}$ such that $B_k=A_k^C$ for $k \in \NN$ (Since $A_k \in \mathcal F$, $A_k^C \in \mathcal F$) \\
			We are done as $\cap_{k \in \NN} B_k=\cup_{k \in \NN} A_k^C \in \mathcal F$ and $\cup_{k \in \NN} A_k=(\cup_{k \in \NN} A_k^C)^C \in \mathcal F$
		\item $A \setminus B = A \cap B^C$ which trivially belongs in $\mathcal F$ by the definition of a $\sigma$-algebra and ii)
	\end{enumerate}
\end{proof}

\begin{lemma}
	Let $F_\alpha$, $\alpha \in I$, be a $\sigma$-algebra over $\Omega$, where $I$ is an arbitrary index set, then $F \coloneqq \cap_{\alpha \in I} F_\alpha$ is also a $\sigma$-algebra
\end{lemma}

\begin{proof}
	Let us verify the properties of a $\sigma$-algebra one by one:
	\begin{enumerate}[i)]
		\item $\Omega \in \mathcal F_\alpha$ $\forall \alpha \in I$ $\implies \Omega \in  \cap_{\alpha \in I} F_\alpha$
		\item $A \in \mathcal F \implies A \in \mathcal F_\alpha$ $\forall \alpha \in I$ and $A \in \mathcal F_\alpha \implies A^C \in \mathcal F_\alpha$ $\forall \alpha \in I$ \\
		Since $A^C \in \mathcal F_\alpha$ $\forall \alpha \in I$, $A^C \in \cap_{\alpha \in I} F_\alpha = \mathcal F$
		\item $A_n \in \mathcal F \implies A_n \in \mathcal F_\alpha$ $\forall n \in \NN,\alpha \in I$ and $A_n \in \mathcal F_\alpha \,\forall n \in \NN \implies \cup_{n \in \NN} A \in \mathcal F_\alpha$ $\forall \alpha \in I$ \\
		Since $\cup_{n \in \NN} A \in \mathcal F_\alpha$ $\forall \alpha \in I$, $\cup_{n \in \NN} A \in \cap_{\alpha \in I} F_\alpha = \mathcal F$
	\end{enumerate}
\end{proof}

\begin{definition}[Generated $\sigma$-algebra]
	Let $A \subseteq 2^\Omega$, then $\displaystyle \sigma(A) \coloneqq \bigcap_{\substack{\mathcal F \in 2^\Omega \\ \mathcal F \text{ is a }\sigma-\text{algebra} \\ A \subseteq \mathcal F}} \mathcal F$
\end{definition}

\begin{definition}[Ring]
	$\mathcal F \subseteq 2^\Omega$ is a ring if:
	\begin{enumerate}[i)]
		\item $\phi \in \mathcal F$
		\item $A,B \in \mathcal F \implies A \setminus B \in \mathcal F$
		\item $A,B \in \mathcal F \implies A \cup B \in \mathcal F$
	\end{enumerate}
\end{definition}

\begin{definition}[Semiring]
	$\mathcal F \subseteq 2^\Omega$ is a semiring if:
	\begin{enumerate}[i)]
		\item $\phi \in \mathcal F$
		\item $A,B \in \mathcal F \implies A \cap B \in \mathcal F$
		\item $A,B \in \mathcal F \implies \exists k \in \mathbb{N}$ such that $\exists \{E_i\}_{i=1}^k \in \mathcal F$ such that $E_i \cap E_j = \phi$ and $\cup_{i=1}^k E_i = B \setminus A$
	\end{enumerate}
\end{definition}

\begin{definition}[Semiring]
	$\mathcal F \subseteq 2^\Omega$ is a semiring if:
	\begin{enumerate}[i)]
		\item $\phi \in \mathcal F$
		\item $A,B \in \mathcal F \implies A \cap B \in \mathcal F$
		\item $A,B \in \mathcal F \implies \exists k \in \mathbb{N}$ such that $\exists \{E_i\}_{i=1}^k \in \mathcal F$ such that $E_i \cap E_j = \phi$ and $\cup_{i=1}^k E_i = B \setminus A$
	\end{enumerate}
\end{definition}

\begin{definition}[$\lambda$-system]
	$\mathcal F \subseteq 2^\Omega$ is a $\lambda$-system if:
	\begin{enumerate}[i)]
		\item $\Omega \in \mathcal F$
		\item $A,B \in \mathcal F$ such that $A \subseteq B \implies B \setminus A \in \mathcal F$
		\item $A_n \in \mathcal F, n \geq 1$ and $A_i \cap A_j = \phi$ $\forall i \neq j$ $\implies \cup_{n \geq 1} A_n \in \mathcal F$
	\end{enumerate}
\end{definition}

\begin{definition}[$\pi$-system]
	$\mathcal F \subseteq 2^\Omega$ is a $\pi$-system if $A, B \in \mathcal F \implies A \cap B \in \mathcal F$
\end{definition}

\begin{lemma}
	Every ring is a semiring
\end{lemma}

\begin{proof}
	Let $\mathcal F \subseteq 2^\Omega$ be a ring. \\
	It is easy to see that $\phi \in \mathcal F$ and $B \setminus A \in \mathcal F$ if $A,B \in \mathcal F$ by definition. \\
	We just need to prove that if $A,B \in \mathcal F$, then $A \cap B \in \mathcal F$ which follows from the fact that $A \cap B = A\setminus (A \setminus B)$
\end{proof}

\begin{lemma}
	Let $\mathcal{F} \subseteq 2^\Omega$ be both a $\lambda$-system and a $\pi$-system. Then $\mathcal F$ is a $\sigma$-algebra
\end{lemma}

\begin{proof}
	Firstly, $\Omega \in \mathcal F$ and $\mathcal F$ is a $\lambda$-system. \\
	Next let $A \in \mathcal F$ and then $A^C = \Omega \setminus A \in \mathcal F$ as $A \subseteq \Omega$ and $A, \Omega \in \mathcal F$ \\
	Finally let $A_n \in \mathcal F$, $n \geq 1$ and subsequently define $B_n=A_n \setminus \cup_{i=1}^{n-1} A_i=A_n \cap (\cup_{i=1}^{n-1} A_i)^C= A_n \cap (\cap_{i=1}^{n-1} A_i^C)$ for $n \in \NN$ \\
	Since we had already shown that $A_n^C \in \mathcal F$ $\forall n \in \NN$ and since $\mathcal F$ is a $\pi$-system, we have $B_n \in \mathcal F$ $\forall n \in \NN$ \\
	Furthermore, $B_n \cap B_m = \phi$ if $n \neq m$ and since $\mathcal F$ is a $\lambda$-system, $\cup_{n \geq 1} A_n=\cup_{n \geq 1} B_n \in \mathcal F$
	\vspace{10pt}
	
	Hence, $\mathcal F$ is a $\sigma$-algebra.
\end{proof}

\section{Lecture 2 (August 14 2024)}

\begin{definition}[Measure and Measurable space]
	For any non-empty set $\Omega$, $(\Omega,\mathcal F)$ is a measurable space for a $\sigma$-algebra $\mathcal F$ over $\Omega$ and $\mu : \mathcal F \to [0,\infty]$ is called a measure on $(\Omega, \mathcal F)$, if:
	\begin{enumerate}[a)]
		\item $\mu(\phi)=0$
		\item $A_n \in \mathcal F$, $n \geq 1$ such that $A_n \cap A_m = \phi$ if $n \neq m \implies \mu(\cup_{n \geq 1} A_n) = \sum_{n \geq 1} \mu(A_n)$
	\end{enumerate}
	If $\mu(Omega)=1$, $\mu$ is also a probability measure
\end{definition}

\begin{theorem}[$\pi$-$\lambda$ Theorem]
	If $\mathcal P,\mathcal L \subseteq 2^\Omega$ such that $\mathcal P$ is a $\pi$-system, $\mathcal L$ is a $\lambda$-system and $\mathcal P \subseteq \mathcal L$, then $\sigma(\mathcal P) \subseteq \mathcal L$
\end{theorem}

\begin{proof}
	Define $\displaystyle \mathcal L_0 \coloneqq \bigcap_{\substack{\mathcal P \subseteq \mathcal L' \subseteq 2^\Omega \\ \mathcal L' \text{ is a $\lambda$-system}}} \mathcal L'$, which is also a $\lambda$-system with $\mathcal P \subseteq \mathcal L_0 \subseteq \mathcal L$
	\vspace{10pt}
	
	In order to show that $\mathcal L_0$ is a $\pi$-system, we define $\mathcal L_A \coloneqq \{B \in \mathcal L_0 \mid A \cap B \in \mathcal L_0\}$ for every $A \subseteq \mathcal L_0$. For every $A \subseteq \mathcal L_0$, notice that $\mathcal L_A \subseteq \mathcal L_0$ by definition. Note that $\mathcal L_A$ is a $\sigma$-algebra since:
	\begin{enumerate}[i)]
		\item $A \cap \Omega=A \in \mathcal L_0 \implies \Omega \in \mathcal L_A$
		\item $\forall B,C \in \mathcal L_A$ such that $B \subseteq C$ \\
		$\implies	(A \cup B),(A \cup C) \in \mathcal L_0$ with $A \cap B \subseteq A \cap C \implies (A \cap C)\setminus (A \cap B) \in \mathcal L_0$
		\begin{align*}
			()A \cap C)\setminus (A \cap B) &= (A \cap C) \cap (A \cap B)^C \\
			&= (A \cap C) \cap (A^C \cup B^C) \\
			&= (A^C \cap A \cap C) \cup (B^C \cap A \cap C) \\
			&= A \cap (C \setminus B) \in \mathcal L_0 \\
			&\implies C\setminus B \in \mathcal L_A
		\end{align*}
		\item $A_n \in \mathcal L_A$, $n \geq 1$ such that $A_n \cap A_m = \phi$ for $n \neq m$ \\
		$\implies A_n \cap A \in \mathcal L_0$ such that $(A_n \cap A) \cap (A_m \cap A) = \phi$ for $n \neq m$ \\
		$\implies \cup_{n \geq 1} (A_n \cap A) = A \cap (\cup_{n \geq 1} A_n) \in \mathcal L_0 \implies \cup_{n \geq 1} A_n \in \mathcal L_A$
	\end{enumerate}
	
	If $\forall A \in \mathcal L_0$, $\mathcal P \subseteq \mathcal L_A$ then we are done \\
	Else if $A \in \mathcal P$, let $B \in \mathcal P$
	\begin{align*}
		A \cap B \in \mathcal P \subseteq \mathcal L_0 &\implies A \cap B \in \mathcal L_0 \\
		&\implies B \in \mathcal L_0 \\
		&\implies P \subseteq \mathcal L_0 \\
		&\implies \mathcal L_A = \mathcal L_0
	\end{align*}
	Else if  $A \in \mathcal L_0 \setminus \mathcal P$, let $B \in \mathcal P$
	\begin{align*}
		\mathcal L_b = \mathcal L_0 &\implies A \cap B \in \mathcal L_0 \\
		&\implies B \in \mathcal L_A \\
		&\implies \mathcal P \subseteq \mathcal L_A
	\end{align*}
\end{proof}

\begin{definition}[Topological space]
	$(\Omega, \tau)$ ($\tau \subseteq 2^\Omega$) if:
	\begin{enumerate}[i)]
		\item $\Omega, \phi \in \tau$
		\item $O_n \in \tau$ for $n \in I$ $\implies$ $\cup_{n \in I} O_n \in \tau$
		\item $O_i \in \tau$ for $1 \leq i \leq n$ where $n \in \NN$ $\implies$ $\cap_{i=1}^n O_i \in \tau$
	\end{enumerate}
\end{definition}

\begin{definition}[Borel $\sigma$-algebra]
	For a topological space $(\Omega, \tau)$, $\sigma(\tau)$ is called the Borel $\sigma$-algebra for the topological space (often denoted as $\mathcal B(\Omega)$)
\end{definition}

\begin{example}
	[$\mathcal B(\RR^n)$]
	$\mathcal B(\RR^n)$ is the Borel $\sigma$-algebra for $\RR^n$ under the usual topology, i.e. $\mathcal B(\RR^n)=\sigma(\text{Open subsets of }\RR^n,\text{usual topology})$
\end{example}