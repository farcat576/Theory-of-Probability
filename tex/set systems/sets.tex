\chapter{Sets}
\label{ch:sets}

\section{Lecture 0}

If we are to believe Wikipedia blindly, then
\begin{quote}
	In mathematics, a set is a collection of different things; these things are called elements or members of the set and are typically mathematical objects of any kind: numbers, symbols, points in space, lines, other geometrical shapes, variables, or even other sets.
\end{quote}

\begin{example}
	[Famous sets]
	Some sets you must have seen by now:
	\begin{enumerate}
	\item Natural Numbers: $\NN$
	\item Integers: $\ZZ$
	\item Rational Numbers: $\QQ$
	\item Real Numbers: $\RR$
	\item Complex Numbers: $\CC$
	\end{enumerate}
\end{example}

\begin{abuse}
	We have progressed from the age of Bhimbetka. So no proof by pictures, Venn diagrams, or any other form of
	doodling. You are however encouraged to draw pictures to gain intuition.
\end{abuse}

Assuming one has done high school mathematics or gone through the required \hyperref[ch:sets_functions]{notation to be followed}, we will refresh our set theory basics with a couple of exercises

\section{\problemhead}

\begin{problem}
	 Define the symmetric difference between two sets as $A \Delta B \triangleq(A \backslash B) \cup(B \backslash A)$. Show that:
	 \begin{enumerate}[a)]
		\item $A \Delta B=(A \cup B) \backslash(A \cap B)$.
		\item $A \backslash(A \Delta B)=A \cap B$.
	\end{enumerate}
	\begin{hint}
		Tbdl
	\end{hint}
	\begin{sol}
		Tbdl
	\end{sol}
\end{problem}

\begin{problem}
	If $A \subseteq B$ and $C \subseteq D$, then prove that $(A \cap C) \cup(B \cap D)=B \cap D$.
\end{problem}

\begin{problem}
	Let $I$ be any set, and consider the collection of sets $\left\{A_{i}: i \in I\right\}$. Show that for any other set $B, B \cap\left(\cap_{i \in I} A_{i}\right)=\cap \cap_{i \in I}\left(B \cap A_{i}\right)$. Hence prove $B \backslash \cup_{i \in I} A_{i}=\cap_{i \in I}\left(B \backslash A_{i}\right)$.
\end{problem}

\begin{problem}
	Prove the following:
	\begin{enumerate}[a)]
	\item $\cap_{n=1}^{\infty}\left(-\frac{1}{n}, \frac{1}{n}\right)=\{0\}$.
	\item $\cap_{n=1}^{\infty}\left[1-\frac{1}{n}, \infty\right)=[1, \infty)$.
	\end{enumerate}
\end{problem}

\begin{problem}
	Consider the function $f: X \rightarrow Y$, and let $A \subseteq X$ and $B \subseteq Y$. Define the generalised inverse of $f$ by $f^{-1}(B)=\{x \in$ $X: f(x) \in B\}$. Also, define $f(A)=\{y \in Y: \exists x \in A$ such that $f(x)=y\}$.
	\begin{enumerate}[a)]
	\item Show that $f^{-1}\left(B_{1} \cup B_{2}\right)=f^{-1}\left(B_{1}\right) \cup f^{-1}\left(B_{2}\right), \forall B_{1}, B_{2} \subseteq Y$.
	\item Show that $f^{-1}\left(B_{1} \cap B_{2}\right)=f^{-1}\left(B_{1}\right) \cap f^{-1}\left(B_{2}\right), \forall B_{1}, B_{2} \subseteq Y$.
	\item Show that $A \subseteq f^{-1}(f(A))$. Give an example to show that the inclusion can be strict.
	\item Show that $f\left(f^{-1}(B)\right) \subseteq B$. Show that equality holds if $f$ is surjective.
	\end{enumerate}
\end{problem}

\begin{problem}
	Consider a function $f: \mathbb{R}^{n} \rightarrow \mathbb{R}^{n}$. Let $A_{i} \subseteq \mathbb{R}, 1 \leq i \leq n$.
	\begin{enumerate}[a)]
	\item Show that $f\left(\times_{i=1}^{n} A_{i}\right) \subseteq f\left(A_{1} \times \mathbb{R} \times \ldots \times \mathbb{R}\right) \cap f\left(\mathbb{R} \times A_{2} \times \ldots \times \mathbb{R}\right) \cap \ldots \cap f\left(\mathbb{R} \times \mathbb{R} \times \ldots \times A_{n}\right)$.
	\item Provide an example to show that the inclusion can be strict.
	\end{enumerate}
\end{problem}

\begin{problem}[Final boss: (Bradley, if you happen to be a Fullmetal Alchemist fan)]
	Define the set $T_{n} \triangleq \cup_{i=0}^{n-1}\left[\frac{i}{n}, \frac{i+1}{n}\right] \times\left[0,1-\frac{i}{n}\right]$. Show that $\cap_{n \in \mathbb{N}} T_{n}=\left\{(x, y) \in \mathbb{R}^{2}: x+y \leq 1, x \geq 0, y \geq 0\right\}$.
\end{problem}